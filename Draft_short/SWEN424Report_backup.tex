%% $RCSfile: proj_proposal.tex,v $
%% $Revision: 1.2 $
%% $Date: 2010/04/23 02:40:16 $
%% $Author: kevin $

\documentclass[11pt, a4paper, oneside, openright]{article}

\usepackage{float} % lets you have non-floating floats
\usepackage{url} % for typesetting urls
\usepackage{graphicx} % for including images

\newfloat{fig}{thp}{lof}[section]
\floatname{fig}{Figure}


\title{A Domain-Customised Business Process Model for Postal Services}
\author{Ashleigh Cains (300198169) \\ Alexandre Sawczuk da Silva (300193950)}

\usepackage[font,ecs]{vuwproject}

% Unless you've used the bschonscomp or mcompsci
%  options above use
   \otherdegree{SWEN424 --- Model-Driven Development}
% here to specify degree

% Comment this out if you want the date printed.
%\date{}

\begin{document}

% Make the page numbering roman, until after the contents, etc.
\frontmatter

%%%%%%%%%%%%%%%%%%%%%%%%%%%%%%%%%%%%%%%%%%%%%%%%%%%%%%%

\begin{abstract}
This report details a project which explored the concept of Model-driven
Development (MDD). This was done by creating a visual language that enabled the
modelling of business processes in a mail delivery company. To that end, a tool
called Generic Modelling Environment (GME) was used, and a metamodel describing
the language was created. Subsequently, the sample business process of
requesting a delivery was modelled in the newly created language. Finally, an
interpreter was developed and the model was translated into a Java structure
using the interface provided by GME. The proof-of-concept developed in this
project demonstrates the usefulness of the MDD paradigm, which enables greater
involvement of non-technical users and  leads to faster production of code.
\end{abstract}

%%%%%%%%%%%%%%%%%%%%%%%%%%%%%%%%%%%%%%%%%%%%%%%%%%%%%%%

\maketitle

%\tableofcontents

% we want a list of the figures we defined
%\listof{fig}{Figures}

%%%%%%%%%%%%%%%%%%%%%%%%%%%%%%%%%%%%%%%%%%%%%%%%%%%%%%%

\mainmatter

%%%%%%%%%%%%%%%%%%%%%%%%%%%%%%%%%%%%%%%%%%%%%%%%%%%%%%%

\section{Introduction}

This report presents the worked performed and the knowledge gained during the
team project for SWEN424, the Software Engineering course on Model-Driven
Development offered by Victoria University of Wellington in 2013. In this
project, the team developed the proof-of-concept for a visual modelling language
intended to be applied when creating Business Process Models (BPMs) for the
domain of a mail delivery company.

First the concepts of Model-Driven Development and Business Process Models
are introduced. Then the business case used as the basis for this project is
explained and the development process of the proof-of-concept is examined, with
a focus on the modelling tool employed. Finally, the conclusions reached by the
team about the development process are laid out.


\section{Model-Driven Development}

Before the project can be discussed, the concept of Model-Driven development
must be explained. Model-driven development is the notion that a system can be
described in terms of a model that is later transformed into the system
\cite{mdd}.
Essentially, this allows users to build systems by modelling them in an
intuitive, visual language and then transforming them into a working textual
representation. The transformation step generates some or all of the system
implementation in the target language/representation. When the right balance
between abstraction and simplification is struck, it is possible to reduce the
complexity of the problem being modeled without overlooking any of its critical
aspects, a goal that has motivated computer scientists and software engineers
for many decades \cite{goodBadUgly}.

An invaluable benefit of successful model-driven development is the
automation it allows, which translates to much higher levels of productivity
and more accurate end-results \cite{mddPragmatics}. Specifically, this automation has to do
with the generation of programs from models, as well as the verification of
model correctness. These techniques are growing in maturity and as a result
are being increasingly used by the software engineering community, leading
to the development of industry standards.

\section{Business Process Models}

Another fundamental concept in this project is that of a Business Process Model.
A Business Process Model, often shortened to a BPM, is a graphical way of
displaying the processes utilised by a business. From the simple flow charts
used at the beginning of the 20th century, BPMs later motivated the creation of
a dedicated notation, the BPMN (Business Process modelling and Notation) \cite{bpmSim}.

A BPM is a valuable technique for many reasons, including the following:

\begin{enumerate}
  \item It provides a straightforward visual representation of the processes
  within a business.
  \item The visual notation, in its turn, enables the identification of common
  patterns found in business processes \cite{bpmImp}, leading to better business
  understanding and potentially better decision-making.
  \item BPMs provide stakeholders with a communication tool that does not
  require technical knowledge in order to be used, yet is precise enough to
  convey meaningful specifications to the developers in charge of constructing
  the system.
\end{enumerate}

The point regarding stakeholder communication is especially important. As
shown by Al-Rawas and Eastbrook \cite{communication}, communication problems among stakeholders
are the greatest causes for failures and delays in engineering projects. More
specifically, their study shows that one area of difficulty is the translation
of natural language requirements into some form of representational objects. The
problem is that once the requirements are translated into the new
representation, many of the stakeholders are no longer able to understand the
notations employed by business analysts. It is in this context that the
advantages of utilising models become apparent.

A model that is understandable to non-technical stakeholders allows them to
participate in the process of formalising requirements for business processes.
This first-hand involvement, in its turn, increases the likelihood of creating
specifications that are based on correctly interpreted requirements. Ultimately,
achieving accurate specifications earlier in the analysis period of a project
may lead to increased productivity, since models are likely to need less
reworking.

The benefits of employing BPMs are evidenced by the increased number of
companies across the world seeking to employ this technique as a core
business practice \cite{bpmBenefits}. Despite the necessity to configure BPMs for each
particular business, which can be costly, research is being done towards
less labour-intensive solutions \cite{bpmCorrectness}. BPMs can be applied in a wide variety
of domains, including the one discussed next.


\section{Selected Domain and Idea Being Modelled}

The domain selected for this project was that of a postal company, more
specifically the server-side of the online services provided by the company.
These services are not typically large, so it is quite feasible to model them.
Additionally, they operate on source code that can be generated based on a BPM
model, making them ideal candidates for the Model-Driven Development approach.

The New Zealand Post was chosen as an example company for this project, since it
offers a myriad of services, both online and offline. A visit to the company’s
website \cite{nzPost} reveals that the following online services are offered, among
others:

\begin{itemize}
  \item Online shopping
  \item Requesting a delivery online
  \item Setting up mail redirection
  \item Applying for a Post Office box
  \item Booking an urgent/overnight courier
  \item Sending money within New Zealand
  \item Sending money internationally
  \item Applying for 18+ (proof of age) cards
  \item Submitting United Kingdom passport applications
\end{itemize}

The modelling language created in this project was intended to be applicable to
the definition of any online service offered by a postal company. However, the
proof-of-concept model developed in this project was that of an online delivery
request performed by a post service user. The delivery request process starts by
requesting that the user specify the pickup and drop-off locations for the item
to be delivered. Then, that information is validated and the user proceeds to
pay for the delivery. Once the payment transaction is complete, the system
proceeds to determine which transportation mode should be employed for
delivering the item at the drop-off location (it is assumed that items are
picked up always using the same transportation mode). The final step is to
schedule the delivery according to its transportation mode. Once the domain was
selected for modelling, the next steps were to select an adequate modelling
tool, as discussed in the following section.

\section{Selected Modelling Tool}

The modelling tool selected for this project was the Generic modelling Environment
(GME), developed by the Institute for Software Integrated Systems at Vanderbilt
University (TN, United States). This tool was chosen by the team because it was
perceived to be a stable and well-established program with a wealth of online
documentation and an intuitive user interface. The installation process is
straightforward, as it consists of simply downloading the installer from the GME
website \cite{gmeSite} according to your version of Windows (32 or 64-bit), running it and
following the wizard instructions.

GME provides an environment that enables the customisation of modelling languages
for multiple domains \cite{gmePaper}. This is achieved by first defining metamodels that
encode the domain-specific limitations of a given modelling language, then
defining the actual models choosing the metamodel with the appropriate base
concepts. Any additional constraints that cannot be visually modeled are
represented as Object Constraint Language (OCL) statements, as GME provides
constraint-checking capabilities.

Notably, GME is capable of synthesizing output code from a model due to its
architecture based on COM (Component Object Model) technology, an interface
standard created by Microsoft for software components \cite{gmeJava}. This
language-agnostic interface is supported by major programming languages such as
C++ and Java, enabling different types of code artifacts to be generated from a
GME model according to the user’s needs.

This tool was chosen by the team because it was perceived to be a stable and
well-established program with a wealth of online documentation and an intuitive
user interface. Screenshots showing the functionality of the tool are presented
in the following sections. The installation process is straightforward, as it
consists of simply downloading the installer from the GME website \cite{gmeSite}, running
it and following the wizard instructions.

GME is intended to be used for all stages of model-driven development, from the
definition of a modelling language to the creation of a mechanism to translate
models into the desired output. More specifically, GME supports the following
steps:

\begin{itemize}
  \item The creation of a \textit{metamodel}, which establishes the features,
  notation and constraints of the language to be used in the creation of models
  for a given domain. This is accomplished through GME's metamodel perspective.
  \item The creation of a \textit{model}, which is intended to represent a
  conceptual abstraction of a chosen domain. The modelling process can be
  accomplished through GME's model perspective, the first step being the
  selection of the modelling language to be employed (i.e. a metamodel
  previously created using GME).
  \item The creation of an \textit{interpreter}, which is used to convert the
  visual model into a textual representation such as XML, SQL, or a programming
  language. The interpreter is meant to be created using a programming language,
  translating the visual models created in GME through one of the two standard
  interfaces provided (BON or COM -- BON is discussed in more detail later). While GME
  provides the capability of registering an interpreter so that it can be
  invoked from within it, GME itself does not provide a development environment
  for writing the interpreter code.
\end{itemize}

However, before proceeding to the first step (the creation of a metamodel), it
was imperative to select a modelling language. The language was based on the
Business Process Modelling Notation, an industry standard visual language
explained in the next section.

\section{The Business Process Modelling Notation}

BPMN (Business Process modelling and Notation) was selected as the visual
notation for the models in this project. The introductory BPM guide published by
IBM \cite{bpmnIntro} presents the concepts behind this notation in a very structured and
detailed fashion, therefore this section is based on the information
contained in that document.

In the last decade, a non-profit organisation of software industry leaders
called BPMI (Business Process Management Initiative) developed a standard BPM
graphical notation that incorporates elements of traditional flowcharts. The
objective of this standardisation is to unify the understanding of business
analysts and developers involved in this specific field of modelling. Once the
common understanding exists, process owners are capable of following the
evolution of the models and assist in the correction of mistakes.

The visual elements in BPMN were intentionally designed to be similar to a
flowchart diagram, since that is a well-known notation to business analysts and
other modelers. The four types of elements offered by this notation are \textit{flow
objects}, \textit{connecting objects}, \textit{swimlanes} and \textit{artifacts}:

\begin{itemize}
  \item \textbf{Flow objects} are the core objects in a business process, and
  include events (they happen during the course of a business process, and
  affect the process flow), activities (represent the tasks and sub-processes a
  company generally performs) and gateways (control the process flow,
  traditional path decisions, and the merging of paths). Visual representations
  for events, activities and gateways are shown in figures~\ref{fig:eventTypes},
  ~\ref{fig:activityTypes}, and~\ref{fig:gatewayTypes}, respectively.

  \begin{figure}[!ht]
  \centerline{\fbox{\includegraphics[scale=0.15]{eventTypes.png}}}
  \caption{Representation of BPMN event types \cite{bpmWiki}.}
  \label{fig:eventTypes}
  \end{figure}

  \begin{figure}[!ht]
  \centerline{\fbox{\includegraphics[scale=0.5]{activityTypes.png}}}
  \caption{Representation of BPMN activity types \cite{bpmWiki}.}
  \label{fig:activityTypes}
  \end{figure}

  \begin{figure}[!ht]
  \centerline{\fbox{\includegraphics[scale=0.5]{gatewayTypes.png}}}
  \caption{Representation of BPMN gateway types \cite{bpmWiki}.}
  \label{fig:gatewayTypes}
  \end{figure}

  \item \textbf{Connecting objects} are responsible for actually linking the
  flow objects into a process. They can be sequence flows (for showing the order
  of activities), message flows (shows messages exchanged between separate
  business participants), and associations (show input/output of activities).
  Visual representations for connecting objects are shown in figure~\ref{fig:connectionTypes}.

  \begin{figure}[!ht]
  \centerline{\fbox{\includegraphics[scale=0.5]{connectionTypes.png}}}
  \caption{Representation of BPMN connecting object types \cite{bpmWiki}.}
  \label{fig:connectionTypes}
  \end{figure}

  \item \textbf{Swimlanes} enable the organisation of activities into different
  groups, according to their meaning or functionality. They can be pools (basic
  container) or lanes (sub-partitions of pools for further organisation).
  \item \textbf{Artifacts} are diagram extensions that are used depending on the
  context of the process being modeled. They can be data objects (to model
  required data for activities), groups (for documentation purposes), and
  annotations (additional text information or comments).

\end{itemize}

Even though BPMN is compact, the combination of all its standardised elements
results in a rich and expressive modelling language. The time restriction of this
project did not allow for the full coverage of BPMN concepts, therefore the team
concentrated on the utilisation of a subset of this notation which was smaller
but still expressive enough for a proof-of-concept. The next section describes
this subset in more detail.

\section{Selected Notation Subset}

The notation elements presented in this section are a BPMN subset chosen for
this particular project. The meaning of each element type is explained in the
hierarchical outline that follows. The reader might find it useful to refer to
figures~\ref{fig:eventTypes},~\ref{fig:activityTypes},~\ref{fig:gatewayTypes} and
~\ref{fig:connectionTypes} when examining this section:
\\\\

\begin{itemize}
  \item \textbf{Flow Objects}
  \begin{itemize}
    \item \textbf{Event Types}
    \begin{itemize}
      \item \textbf{Start:} acts as a process trigger, indicated by a single
      narrow bordered circle.
      \item \textbf{Intermediate:} represents something that happens between the
      start and end events; is indicated by a double bordered circle. For
      example, a task could flow to an event that throws a message across to
      another pool, where a subsequent event waits to catch the response before
      continuing.
      \item \textbf{End:} represents the result of a process. It is indicated by a
      single thick or bold bordered circle.
    \end{itemize}
    \item \textbf{Activity Types}
    \begin{itemize}
      \item \textbf{Task:} is a unit of work, the job to be performed.
      \item \textbf{Sub-process:} used to hide or reveal additional levels of
      business process detail. Has its own self-contained start and end events;
      flow objects; connecting objects; and artifacts.
      \item \textbf{Transaction:} is a set of activities that logically belong
      together; it might follow a specified transaction protocol (involves
      messaging).
    \end{itemize}
    \item \textbf{Gateway Types}
    \begin{itemize}
      \item \textbf{Exclusive:} when splitting, it routes the sequence flow to
      exactly one of the outgoing branches. When merging, it awaits one incoming
      branch to complete before triggering the outgoing flow.
      \item \textbf{Inclusive:} when splitting, one or more branches are
      activated. All active incoming branches must complete before merging.
      \item \textbf{Event-based:} is always followed by catching events or
      receive tasks. Sequence flow is routed to the subsequent event/task which
      happens first.
      \item \textbf{Exclusive event-based:} each occurrence of a subsequent
      event starts a new process instance.
    \end{itemize}
  \end {itemize}
  \item \textbf{Connecting Objects}
  \begin{itemize}
    \item \textbf{Message Flows:} symbolize information flow across
    organizational boundaries and can be attached to pools, activities, or
    message events.
    \item \textbf{Sequence Flows:} define the order of execution of activities.
  \end{itemize}
\end{itemize}

Some of the elements in the generic hierarchy above were then customised by the
team to meet the necessities of a model that abstracts postal company services.
The following outline presents descriptions only for the customised elements:

\begin{itemize}
  \item \textbf{Event Types}
  \begin{itemize}
    \item \textbf{End}
    \begin{itemize}
      \item \textbf{Error:} a specialised version of the end event was created
      to denote an error in the execution of the process which leads to its
      termination.
    \end{itemize}
  \end{itemize}
  \item \textbf{Activity Types}
  \begin{itemize}
    \item \textbf{Task}
    \begin{itemize}
      \item \textbf{Delivery:} represents the task of scheduling the delivery of
      an item to a certain address. Deliveries offer different transportation
      mode options: by car, by boat, by bike, and by plane.
    \end{itemize}
    \item \textbf{Transaction}
    \begin{itemize}
      \item \textbf{Input:} an input retrieval transaction. It can be as simple
      as retrieving content from one field or as complex as retrieving large
      forms and uploaded documents.
      \item \textbf{Payment:} a payment transaction. It debits a certain amount
      of money from the customer account using the credit card information
      provided.
      \item \textbf{Money Transfer:} a money transfer transaction. Sends a
      certain amount of money from one bank account to another.
      \item \textbf{Third-Party Transaction:} an exchange between the post
      company and a third-party service.
    \end{itemize}
  \end{itemize}
  \item \textbf{Gateway Types}
  \begin{itemize}
    \item \textbf{Exclusive}
    \begin{itemize}
      \item \textbf{Determine Transportation Mode:} determines how an item
      should be transported for delivery.
      \item \textbf{Validate Input:} determines whether the input provided by
      the user was valid.
      \item \textbf{Validate Payment:} determines whether a user payment was
      valid.
      \item \textbf{Validate Money Transfer:} determines whether a money
      transfer was valid.
      \item \textbf{Validate Third-Party Transaction:} determines whether a
      transaction that depends on a third-party service was valid.
    \end{itemize}
  \end{itemize}
\end{itemize}

These notation elements were incorporated into the metamodel, as discussed next.

\section{Creating the Metamodel}
The metamodel was created in GME by starting a new project using the MetaGME
paradigm, which is especially designed for the definition of modelling
languages. It was structured following the BPM notation described in the
previous section, resulting in the class diagram shown in figure~\ref{fig:metamodel}.

\begin{figure}[!ht]
\centerline{\fbox{\includegraphics[scale=0.6]{metamodel.png}}}
\caption{Screenshot BPMN metamodel.}
\label{fig:metamodel}
\end{figure}

The picture shows the metamodel was developed according to the BPM hierarchy
described earlier. Fundamentally, a model contains connections and flow objects.
Connections can either show the general flow of a process or the path of event
messages, while flow objects can be the activities, gateways and events (e.g.
start, end) described earlier. Once developed, the metamodel was interpreted and
registered using the MetaGME interpreter (done by clicking on the cog icon).
This is a necessary step for making this paradigm -- the modelling language --
available in GME for the subsequent creation of models.

\section{Addition of Model Constraints}
In addition to the metamodel, rules can be written to ensure a model's validity.
These rules are expressed in a rich language called OCL (Object Constraint
Language) which is defined as part of the standard UML notation. OCL is based on
sentences that are either true or false and must evaluate to "true" in order to
satisfy a constraint. For this project, the team has included constraints using
GME, as it supports OCL 2.0. An OCL expression becomes a constraint when it is
specified in the metamodel attached to the metaobject it applies to.

To add constraints in GME, the ParadigmSheet that contains the metamodel must be
open. From there the “Constraints” aspect needs to be selected and a constraint
inserted into the metamodel. An association needs to be made between the
constraint and the metaobject by drawing a connection between the two so a
context can be given to the OCL expression. In the attributes panel, a name,
description and the expression can be added to the constraint. From there the
metamodel can be reinterpreted and registered to test the constraint in the
model \cite{gmeSite}.

An example of a constraint used in the project was based on the BPMN rules of:
\begin{enumerate}
  \item A start event cannot have an incoming connecting object.
  \item A start event must have an outgoing connecting object.
\end{enumerate}

These rules can be translated into an OCL constraint attached to the “Start”
metaobject in the metamodel and can be described by this expression:

\begin{center}
	\textit{self.attachingConnections("dst",Start)-\textgreater size==0 \&\& \\
            self.attachingConnections("src",Start)-\textgreater size==1}
\end{center}

In plain english, this expression evaluates to the set of all connections
attached to “Start” objects and have “Start” as the destination of the
connection, must have a size of zero. This satisfies the first of the two BPMN
rules stated above. The second part of the OCL expression states that the set of
all connections attached to “Start” objects and have “Start” as the source of
the connection, must have a size of 1.

This process of adding constraints has been repeated for a subset of BPMN rules
that apply to the selected notation subset discussed earlier. These rules have
been selected to ensure the basic structure of a model based on BMPN is valid.

\section{Creating the Model}

As a proof-of-concept for the customised BPM language, a sample online delivery
request was created. As shown in figure~\ref{fig:model}, it uses transactions for the
retrieval of input and the collection of payment, as well as delivery scheduling
tasks. Decisions on which transport mode to be used when delivering an item are
made using a gateway, and so it the validation of input. If there are any errors
when making these decisions, the process is terminated with an Error event;
otherwise, it successfully reaches the End event.

\begin{figure}[!ht]
\centerline{\fbox{\includegraphics[scale=0.7]{model.png}}}
\caption{Screenshot of online delivery request model.}
\label{fig:model}
\end{figure}

An interesting feature of GME is that it allows custom icons to be used when
representing each entity in a model. This can be accomplished by providing an
icon path and name to each entity through the preferences panel. The icons
presented earlier were incorporated into this model, making it more easily
interpretable.

\section{Model Translation}
Java was selected as the target language for the code implementation of the
customised BPM framework. Classes and interfaces were defined to mirror the BPM
hierarchy created in the metamodel, except for the connections, which were
implemented as references from one object to the next. In order to translate the
GME model into a correspondent Java structure, an interpreter was written.

As mentioned earlier, GME offers the Builder Object Network (BON) interface, a
standardised representation of models that is supported in both C and Java. This
interface enables code access to all entities in a model, as shown in figure~\ref{fig:bon},
including their instance names and attribute values. In order to run an interpreter
from GME and translate the model from the BON interface to the code
representation, it is necessary to register an interpreter. This is accomplished
by ensuring that the main class of the the Java interpreter implements the
BONComponent interface and by running JavaCompRegister.exe. Once that is done, the
interpreter can be run from GME by clicking on the coffee cup icon that will
appear on the top bar.

\begin{figure}[!ht]
\centerline{\fbox{\includegraphics[scale=0.6]{bon.png}}}
\caption{Classes in the Builder Object Network [14].}
\label{fig:bon}
\end{figure}

Implementing the interpreter was generally straightforward but there were some
challenges. For instance, debugging was somewhat challenging because the code
had to be run from within GME, so no sophisticated debugging tools were
available. However, that issue was solved by showing debugging statements using
programmatically-created dialogs. Another difficulty was in the implementation
of the BPM gateways in Java, as they may lead to several different process paths
depending on the execution of the decision logic. This was solved by
implementing gateways to initially hold a list of all possible paths, then
providing setters that get invoked during the execution of the decision logic
and determine which path should actually be taken.

In terms of achievements, the current version of the interpreter supports the
translation of all elements defined in the metamodel. The resulting Java code is
also completely runnable, even though it lacks the actual decision logic. The
interpreter immediately executes the generated Java code after translation, but
this behaviour could be extended to enable the code to be serialized and saved
for later loading and execution.

\section{Critical Evaluation of GME}
Throughout the project the team had a chance to analyse the advantages and
disadvantages of GME from a more informed point of view. One one hand, GME is a
very stable tool and throughout the project not a single bug was observed in it. The
user interface is well organised and responsive, and the concept of dragging
elements from the left-hand panel into the modelling area to create new
instances made model assembly fast and intuitive. Learning how to use
GME can be slightly complicated, but the tool is well documented in tutorials and
also in the user manual.

On the other hand, GME needs a better integration between the metamodel and the
model. For example, they cannot be simultaneously open in the same instance of
GME, which makes it cumbersome to update inconsistencies or details in the
metamodel that are noticed while creating a model. Another problem experienced
by the team was that the automatic model update that happens as a result of
changes to the metamodel often did not work, which meant that the model had to
be manually updated via an XML importing operation. This turned out to be a
nuisance even when performing relatively small metamodel changes.

There were several websites for the different versions of GME, which meant that
it was quite easy to accidentally refer to outdated documentation. It would be
nice if those responsible for the GME project could group all of the
documentation and downloads under the same website, organised by version.
Finally, the customisation of icons did not work for the team, even after
carefully following the tutorial. The mechanism for incorporating
icons into the metamodel should be improved, since the team had to resort to
manually setting the icons for each entity type in the model. Overall, the team
considered GME to be a reliable and flexible modelling tool which offers good
features and is intuitive to use, despite some minor inconveniences.

\section{Critical Evaluation of Project Results}
In this section the team insights regarding the work performed in this project
are presented, organised by deliverable:

\begin{itemize}
  \item \textbf{Metamodel:} the metamodel produced is a good starting point for a
  customised BPM language, but further analysis of the business model of a
  postal company is needed to develop a comprehensive solution. This analysis, however,
  would depend on having access to the inner workings of a large postal company
  such as NZ post. Another aspect of the metamodel which needs more work is the
  implementation of constraints. Even though basic constraints were added to the
  metamodel, more are required to guarantee that the order and combination
  of BPM elements always makes sense.
  \item \textbf{Model:} the model produced was expressive, but also a simplified version
  of what a real-life delivery request would entail. For instance, the current
  model does not recover from errors in validation or when choosing a
  transportation mode, instead ending the process with an error. In a real process,
  there should be recovery routines to provide the user with an opportunity to
  correct any problems. Another limitation of the model is that it exists in
  isolation, when in reality each process is likely to communicate with others
  by using the messaging capabilities offered by a BPM.
  \item \textbf{Java code:} the Java code created to represent the BPM is a flexible and
  extensible framework, but it is currently void of real functionality. This structure
  would have to be extended to support the needs of a
  real business process. Additionally, since this framework is meant to
  represent a service that is offered online, its design should incorporate some
  elements of web technologies. It must be noted that any major redesign of
  the Java BPM implementation would incur a redesign of the
  interpreter as well.
\end{itemize}

\section{Work Distribution in Team}
The work distribution of this project was even, with both team members
participating in most tasks. The following table documents which team member has
participated in which project activity:

\begin{table}[H]
\begin{tabular}{|l|l|}
\hline
\multicolumn{1}{|c|}{\textbf{Activity}}                    & \multicolumn{1}{c|}{\textbf{Responsible Team Member}} \\
\hline
Decision of tool, language and concept to be modelled.     & Ashleigh and Alex                                    \\
\hline
Design of metamodel.                                       & Ashleigh and Alex                                    \\
\hline
Creation of metamodel.                                     & Ashleigh                                             \\
\hline
Design and implementation of OCL constraints in metamodel. & Ashleigh                                             \\
\hline
Creation of model.                                         & Alex                                                 \\
\hline
Design and creation of Java code.                          & Alex                                                 \\
\hline
Implementation of interpreter.                             & Alex                                                 \\
\hline
Writing and editing of report.                             & Ashleigh and Alex									  \\
\hline
\end{tabular}
\end{table}

\section{Conclusion}
This project clearly shows that model-driven development is useful because
it enables non-technical users to directly record their knowledge of business
requirements details using a standardised and non-technical language. Since each
entity in the model corresponds to an already-implemented customised
functionality module, the use of a BPM reduces the amount of code that has to be
rewritten each time a process is created and updated.

By its very definition, the efficacy of the model-driven development
approach is dependent on the flexibility of the modelling tool used. While
GME is a good tool, it is geared towards users who understand the abstract
concepts and intricacies of the field of modelling. This means that there
exist opportunities for creating a tool that is geared towards non-technical users,
but which also integrates with the work of technical users. Such a tool would be
commercially viable, as it would improve stakeholder communication and
reduce the amount of work needed during development.

The deliverables produced for this project are part of a proof-of-concept,
but the framework built in Java could easily be populated with complex
logic. Additionally, the Java data structure produced by the translation process
is currently immediately executed, but this behaviour could be modified by adding the
ability to save it for later execution. This project
illustrates the relevance of the model-driven development paradigm, and evidences the
productivity gains to be had from applying this technique to complex systems.


%%%%%%%%%%%%%%%%%%%%%%%%%%%%%%%%%%%%%%%%%%%%%%%%%%%%%%%
\backmatter
%%%%%%%%%%%%%%%%%%%%%%%%%%%%%%%%%%%%%%%%%%%%%%%%%%%%%%%

%\bibliographystyle{ieeetr}
\bibliographystyle{acm}
\bibliography{SWEN424ReportBibliography}
\end{document}
